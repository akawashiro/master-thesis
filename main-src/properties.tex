% !TEX root = ../main.tex

\section{Properties of \LMD \label{sec:properties}}

In this section, we show the basic properties of \LMD: preservation, strong
normalization, confluence for full reduction, and progress for staged
reduction.

% Substitution Lemma

The Substitution Lemma in \LMD{} is a little more complicated than usual
because there are eight judgment forms and two kinds of substitution.
The Term Substitution Lemma states that term substitution $[z \mapsto M]$
preserves derivability of judgments. The Stage Substitution Lemma
states similarly for stage substitution $[\alpha\mapsto A]$.

We let $\mathcal{J}$ stand for the judgments $K \iskind @A$, $\tau::K@A$,
$M:\tau@A$, $K \E J @ A$, $\tau \E \sigma @ A$, and
$M \E N : \tau @ A$.  Substitutions $\mathcal{J}[z \mapsto M]$ and
$\mathcal{J}[\alpha \mapsto A]$ are defined in a straightforward
manner.  Using these notations, the two substitution lemmas are stated as follows:

We proved the next two leammas by simultaneous induction on derivations.

\begin{lemma}[Term Substitution]
  If $\G, z:\xi@B, \D \V \mathcal{J}$ and $\G\V N:\xi @B$, then $\G, (\D[z \mapsto N]) \V \mathcal{J}[z \mapsto N]$.  Similarly, if $\V \G, z:\xi@B, \D$ and
  $\G\V N:\xi @B$, then $\V \G, (\D[z \mapsto N])$.
\end{lemma}

\begin{lemma}[Stage Substitution]
	If $\G \V \mathcal{J}$, then $\G[\beta\mapsto B] \V \mathcal{J}[\beta\mapsto B]$.  Similarly, if $\V \G$, then $\V \G[\beta\mapsto B]$.
\end{lemma}

% Inversion Lemma

The following Inversion Lemma is needed to prove the main theorems.
As usual~\cite{TAPL}, the Inversion Lemma enables us to infer the types of subterms of a term from the shape of the term.

\begin{lemma}[Inversion]\ 
	\begin{enumerate}
		\item If $\G \V (\lambda x:\sigma.M) : \rho$ then there are $\sigma'$ and $\tau'$ such that
		      $\rho = \Pi x:\sigma'.\tau'$, $\G \V \sigma \E \sigma'@A$ and $\G ,x:\sigma'@A\V M:\tau'@A$.
		\item If $\G \V \TB_\alpha M : \tau@A$ then 
		      there is $\sigma$ such that $\tau = \TW_\alpha \sigma$ and $\G \V M : \sigma@A$.
		  \item If $\G \V \Lambda\alpha.M : \tau$ then 
		  there is $\sigma$ such that $\sigma = \forall\alpha.\sigma$ and $\G \V M : \sigma@A$.% and $\alpha \notin \FTV(\G) \cup \FV(A)$.
	\end{enumerate}
\end{lemma}

\begin{proof}
  Each item is strengthened by statements about type equivalence.
  For example, the first statement is augmented by
  \begin{quotation}
    If $\G \V \rho \E (\Pi x:\sigma.\tau) : K @A$, then there exist
    $\sigma'$ and $\tau'$ such that $\rho = \Pi x:\sigma'.\tau'$ and
    $\G \V \sigma \E \sigma' : K @A$ and
    $\G, x:\sigma@A\V \tau \E \tau' : J @A$.
  \end{quotation}
  and its symmetric version.  Then, they are proved simultaneously by induction on derivations.
  Similarly for the others.
  \qed
\end{proof}


% Preservation

Thanks to Term/Stage Substitution and Inversion, we can prove Preservation.
First, we prove Preservation on Level 0 Reduction then prove general
Preservation using it.

\begin{lemma}[Preservation on Level 0 Reduction]
    If \( \G \V M : \tau @ A \) and \( M \longrightarrow_0 M' \), then \( \G \V M' : \tau @ A \).
\end{lemma}

\begin{proof}
    First, there are three cases for $M \longrightarrow_0 M'$.  They are $M
    \longrightarrow_\beta M'$, $M \longrightarrow_\Lambda M'$, and $M
    \longrightarrow_\blacklozenge M'$.  For each case, we can use
    straightforward induction on typing derivations.
\end{proof}

\begin{theorem}[Preservation]
    If \( \G \V M : \tau @ A \) and \( M \longrightarrow M' \), then \( \G \V M' : \tau @ A \).
\end{theorem}

\begin{proof}
    By induction on \( \G \V M : \tau @ A \), all level \( n \) subterms are
    typed. Then, it is a corollary of Preservation on Level 0 Reduction.
\end{proof}

% Strong Normalization

Strong Normalization is also an important property, which guarantees that no
typed term has an infinite reduction sequence.  Following standard proofs (see,
e.g., \cite{harper1993framework}), we prove this theorem by translating \LMD to
the simply typed lambda calculus. As with Preservation, we prove Strong
Normalization on Level 0 Reduction first.

\begin{lemma}[Strong Normalization of Level 0 Reduction]
    \label{lemma:StrongNormalizationofLevel0Reduction}
    If \( \G\V M:\tau@A \) then there is no infinite sequence of terms $(M_i)_{i\ge1}$ and $M_i \longrightarrow_{0} M_{i+1}$ for $i\ge 1$
\end{lemma}

\begin{proof}
    If there is an infinite reduction sequence in $\lambda^{\text{MD}}$
    then there are infinite beta reductions in the sequence.
    This is because reductions other than $\beta$-reduction reduce the size of a term.

    Then, we show Strong Normalization by proof by contradiction.
    We assume that there is an infinite reduction in a typed \LMD from typed term $M$.
    We can construct a typed term of simply typed lambda calculus $\natural(M)$ from Preservation of Typing in $\natural$ and
    $\natural(M)$ has infinite reductions from Preservation of Reduction in $\natural$.
    However, $\lambda^\to$ has a property of Strong Normalization, so there is no infinite reductions.
\end{proof}

\begin{lemma}[Strong Normalization of Level $n$ Reduction]
    \label{lemma:StrongNormalizationofLevelnReduction}
    \( n \) is a arbitary integer which is greater than 0.
    If \( \G\V M:\tau@A \) then there is no infinite sequence of terms $(M_i)_{i\ge1}$ and $M_i \longrightarrow_n M_{i+1}$ for $i\ge 1$.
\end{lemma}

\begin{proof}
    If \( \G\V M:\tau@A \) then there are \( \G', \sigma \) and \( B \) such that \( \G' \V N : \sigma @ B \) for any level \( n \) subterm \( N \). This is because \( N \) is embedded into \( M \) using \KApp or \TApp. Then, \( N \) is strong nomalizing because of Lemma \ref{lemma:StrongNormalizationofLevel0Reduction}.
\end{proof}

\begin{theorem}[Strong Normalization]
    \label{theorem:StrongNormalization}
    If \( \G\V M:\tau@A \) then there is no infinite sequence of terms $(M_i)_{i\ge1}$ and $M_i \longrightarrow M_{i+1}$ for $i\ge 1$
\end{theorem}

\begin{proof}
    Maximum value of levels of subterms of \( M \) is finite because the size
    of \( M \) is finite. So the maximum length of \( \{ M_i \} \) is finite
    from Lemma \ref{lemma:StrongNormalizationofLevel0Reduction} and
    \ref{lemma:StrongNormalizationofLevelnReduction}.
\end{proof}

Confluence is a property that any reduction sequences from one typed term
converge.  Since we have proved Strong Normalization, we can use Newman's
Lemma~\cite{DBLP:books/daglib/0092409} to prove Confluence.

\begin{theorem}[Confluence]
	For any term $M$, if $M \longrightarrow^* M'$ and $M \longrightarrow^* M''$ then
	there exists $M'''$ that satisfies $M' \longrightarrow^* M'''$ and $M'' \longrightarrow^* M'''$.
\end{theorem}

\begin{proof}
  We can easily show Weak Church-Rosser. Use Newman's Lemma.
	% Because we proved Strong Normalization of \LMD, 
	% we can use Newman's lemma to prove Confluence of \LMD.
	% Then, what we must show is Weak Church-Rosser Property now.
	% When we consider two different redexes in a \LMD term, they can only be disjoint, or one is a part of the other.
	% In short, they are never overlapped each other.
	% So, we can reduce one of them after we reduce another.
\end{proof}

Now, we turn our attention to staged semantics.  First, the staged
reduction relation is a subrelation of full reduction, so Subject
Reduction holds also for the staged reduction.

\begin{theorem}
  If $M \longrightarrow_s M'$, then $M \longrightarrow M'$.
\end{theorem}
\begin{proof}
  Easy.
\end{proof}

The following theorem Unique Decomposition ensures that every typed
term is either a value or can be uniquely decomposed to an evaluation
context and a redex, ensuring that a well-typed term is not
immediately stuck and the staged semantics is deterministic.

\begin{theorem}[Unique Decomposition]
  If $\G$ does not have any variable declared at stage $\varepsilon$ 
  and $\G \V M : \tau @ A$, then either
  \begin{enumerate}
  \item $ M \in V^A$, or
  \item $M$ can be uniquely decomposed into an evaluation context and a redex, that is, there uniquely exist $B, E^A_B$, and $R^B$ such that $M = E^A_B[R^B]$.
  \end{enumerate}
\end{theorem}

\begin{proof}
  We can prove by straightforward induction on typing derivations.
\end{proof}

The type environment $\G$ in the statement usually has to be empty;
in other words, the term has to be closed.  The condition is relaxed here
because variables at stages higher than \(\varepsilon\) are considered
symbols.  In fact, this relaxation is required for proof by induction
to work.

% This theorem is important because it guarantees
% that the evaluation context decides a redex to reduce deterministically.
% Specifically speaking, this theorem guarantee that 
% when you write a interpreter using the evaluation context of \LMD,
% your interpreter works just as intended.
      
Progress is a corollary of Unique Decomposition.

\begin{corollary}[Progress]
	If $\G$ does not have any variable declared at stage $\varepsilon$ and $\G \V M : \tau  @ A$, then
	$ M \in V^A $ or there exists $M'$ such that $M \longrightarrow_s M'$.
\end{corollary}

