% !TEX root = ../main.tex

\section{Formal Definition of \LMD \label{sec:formal-definition}}
\label{sec:formal}

In this section, we give a formal definition of \LMD, including
the syntax, full reduction, and type system.  In addition to the full reduction,
in which any redex at any stage can be reduced, we also give staged reduction,
which models program execution (at \(\varepsilon\)-stage).

\subsection{Syntax}

We assume the denumerable set of \emph{type-level constants}, ranged over by
metavariables \(X, Y, Z\), the denumerable set of \emph{variables}, ranged
over by \(x,y,z\), the denumerable set of \emph{constants}, ranged over by
\(c\), and the denumerable set of \emph{stage variables}, ranged over by
\(\alpha, \beta, \gamma\).  The metavariables \(A, B, C\) range over
sequences of stage variables; we write \(\varepsilon\) for the empty
sequence. \LMD is defined by the following grammar:

{%\small
\begin{align*}
   \textrm{kinds}             &  & K,J,I,H,G                & ::= * \mid \Pi x:\tau.K                                                           \\
    \textrm{types}             &  & \tau,\sigma,\rho,\pi,\xi & ::= X \mid \Pi x:\tau.\sigma \mid \tau\ M \mid \TW_{\alpha} \tau \mid \F\alpha.\tau \\
    \textrm{terms}             &  & M,N,L,O,P                & ::= c \mid x \mid \lambda x:\tau.M\ \mid M\ N \mid \TB_\alpha M                   \\
                               &  &                          & \ \ \ \ \mid \TBL_\alpha M \mid \Lambda\alpha.M \mid M\ A \mid \%_\alpha M        \\
    \textrm{signatures}         &  & \Sigma                   & ::= \emptyset \mid \Sigma, X::K \mid \Sigma, c:\tau                               \\
    \textrm{type env.} &  & \Gamma                   & ::= \emptyset \mid  \Gamma,x:\tau @A                                              \\
\end{align*}
}

% Kinds

A kind, which is used to classify types, is either $*$, the kind of
proper types (types that terms inhabit), or $\Pi x\colon\tau.K$, the kind
of type operators that takes $x$ as an argument of type $\tau$ and returns a type
of kind $K$.

% Types

A type is a type-level constant $X$, which is declared in the signature with its kind, a dependent function type $\Pi x:\tau.\sigma$,
an application $\tau\ M$ of a type (operator of $\Pi$-kind) to a term, a code type $\TW_\alpha \tau$, or an $\alpha$-closed type $\F\alpha.\tau$.
An example of an application of a type (operator) of $\Pi$-kind to a term is $\text{Vector}\ 10$; it is well kinded
if, say, the type-level constant $\text{Vector}$ has kind $\Pi x:\I.*$.
A code type $\TW_\alpha \tau$ is for a code fragment of a term of type $\tau$.
An $\alpha$-closed type, when used with $\TW_\alpha$, represents runnable code.

% Terms

Terms include ordinary (explicitly typed) \(\lambda\)-terms, constants,
whose types are declared in signature $\Sigma$, and the following five forms
related to multi-stage programming:
$\TB_\alpha M$ represents a code fragment; $\TBL_\alpha M$ represents escape;
$\Lambda\alpha.M$ is a stage variable abstraction;
$M\ A$ is an application of a stage abstraction $M$ to stage $A$; and
$\%_\alpha M$ is an operator for cross-stage persistence.

% Signature

We adopt the tradition of \LLF-like systems, where types of constants and
kinds of type-level constants are globally declared in a signature $\Sigma$,
which is a sequence of declarations of the form $c:\tau$ and $X::K$. For
example, when we use Boolean in \LMD, $\Sigma$ includes $\B :: *,
\textrm{true}:\B, \textrm{false}:\B$. Type environments are sequences of
triples of a variable, its type, and its stage. We write
\(\textit{dom}(\Sigma)\) and \(\textit{dom}(\Gamma)\) for the set of
(type-level) constants and variables declared in \(\Sigma\) and \(\Gamma\),
respectively. As in other multi-stage
calculi~\cite{taha2003environment,Tsukada,Hanada2014}, a variable declaration
is associated with a stage so that a variable can be referenced only at the
declared stage. On the contrary, constants and type-level constants are
\emph{not} associated with stages; so, they can appear at any stage. We
define well-formed signatures and well-formed type environments later.

The variable $x$ is bound in $M$ by $\lambda x:\tau.M$ and in $\sigma$ by $\Pi
x:\tau.\sigma$, as usual; the stage variable $\alpha$ is bound in $M$ by
$\Lambda \alpha.M$ and $\tau$ by $\F\alpha.\tau$.  The notion of free variables
is defined in a standard manner.  We write $\FV(M)$ and $\FTV(M)$ for the set
of free variables and the set of free stage variables in $M$, respectively.
Similarly, $\FV(\tau)$, $\FTV(\tau)$, $\FV(K)$, and $\FTV(K)$ are defined.  We
sometimes abbreviate $\Pi x:\tau_1.\tau_2$ to $\tau_1 \rightarrow \tau_2$ if
$x$ is not a free variable of $\tau_2$.  We identify $\alpha$-convertible terms
and assume the names of bound variables are pairwise distinct.

The prefix operators $\TW_\alpha, \TB_\alpha, \TBL_\alpha$, and $\%_\alpha$ are
given higher precedence over the three forms $\tau\ M$, $M\ N$, $M\ A$ of
applications, which are left-associative. The binders $\Pi$, $\forall$, and
$\lambda$ extend as far to the right as possible.  Thus, $\F\alpha.\TW_{\alpha}
(\Pi x:\I.\text{Vector}\ 5)$ is interpreted as $\F\alpha.(\TW_{\alpha} (\Pi
x:\I.(\text{Vector}\ 5)))$; and $\Lambda\alpha.\lambda x:\I.\TB_\alpha x\ y$
means $\Lambda\alpha.(\lambda x:\I.(\TB_\alpha x)\ y)$.

\paragraph{Remark:} Basically, we define \LMD to be an extension of
\LTP with dependent types.  One notable difference is that \LMD has
only one kind of \(\alpha\)-closed types, whereas \LTP has two kinds
of \(\alpha\)-closed types \(\forall\alpha.\tau\) and
\(\forall^\varepsilon\alpha.\tau\).  We have omitted the first kind,
for simplicity, and dropped the superscript $\varepsilon$ from the second. It
would not be difficult to recover the distinction to show properties related
to program residualization~\cite{Hanada2014}, although they are left as 
conjectures.

\subsection{Reduction}

Next, we define full reduction for \LMD.  Before giving the definition of
reduction, we define two kinds of substitutions.  Substitution $M[x\mapsto N],
\tau[x \mapsto N]$ and $K[x \mapsto N]$ are ordinary capture-avoiding
substitution of term $N$ for $x$ in term $M$, type $\tau$, and kind $K$,
respectively, and we omit their definitions here.  Substitution $M[\alpha
\mapsto A], \tau [\alpha \mapsto A], K[\alpha \mapsto A]$ and $B[\alpha\mapsto
A]$ are substitutions of stage $A$ for stage variable $\alpha$ in term $M$,
type $\tau$, kind $K$, and stage $B$, respectively.  We show representative
cases below.

{%\small
\begin{align*}
    (\lambda x:\tau.M)[\alpha \mapsto A] & = \lambda x:(\tau[\alpha \mapsto A]).(M[\alpha \mapsto A])                                  \\
    (M\ B)[\alpha \mapsto A]             & = (M[\alpha \mapsto A])\ B[\alpha\mapsto A]                                                 \\
    (\TB_\beta M)[\alpha \mapsto A]      & = \TB_{\beta[\alpha \mapsto A]}M[\alpha \mapsto A]                                          \\
    (\TBL_\beta M)[\alpha \mapsto A]     & = \TBL_{\beta[\alpha \mapsto A]}M[\alpha \mapsto A]                                         \\
    (\%_\beta M)[\alpha \mapsto A]       & = \%_{\beta[\alpha \mapsto A]}M[\alpha \mapsto A]                                           \\
    (\beta B)[\alpha \mapsto A]          & = \beta (B[\alpha\mapsto A])                               & (\text{if } \alpha \neq \beta) \\
    (\beta B)[\alpha \mapsto A]          & = A (B[\alpha\mapsto A])                                   & (\text{if } \alpha = \beta)
\end{align*}
}

Here, $\TB_{\alpha_1\cdots\alpha_n} M$, $\TBL_{\alpha_1\cdots\alpha_n} M$, and
$\%_{\alpha_1\cdots\alpha_n} M$ $(n \geq 0)$ stand for $\TB_{\alpha_1} \cdots
\TB_{\alpha_n} M$, $\TBL_{\alpha_n}\cdots \TBL_{\alpha_1} M$, and
$\%_{\alpha_n}\cdots \%_{\alpha_1} M$, respectively.  In particular,
$\TB_{\varepsilon} M = \TBL_{\varepsilon} M = \%_{\varepsilon} M = M$.  Also,
it is important that the order of stage variables is reversed for $\TBL$ and
$\%$.  We also define substitutions of a stage or a term for variables in type
environment $\G$.

\begin{definition}[Reduction]
    The relations $M \longrightarrow_\beta N$, $M \longrightarrow_\blacklozenge N$, and $M \longrightarrow_\Lambda N$
    are the least compatible relations including terms in types closed under the rules below.
    \begin{align*}
        (\lambda x:\tau.M) N & \longrightarrow_\beta M[x \mapsto N]         \\
        \TBL_\alpha \TB_\alpha M & \longrightarrow_\blacklozenge M          \\
        (\Lambda \alpha.M)\ A & \longrightarrow_\Lambda M[\alpha \mapsto A]
    \end{align*}
\end{definition}

We write $ M \longrightarrow M'$ iff $ M \longrightarrow_\beta M'$, $ M
\longrightarrow_\blacklozenge M'$, or $ M \longrightarrow_\Lambda M'$ and we
call $\longrightarrow_\beta$, $\longrightarrow_\blacklozenge$, and
$\longrightarrow_\Lambda$ $\beta$-reduction, $\blacklozenge$-reduction, and
$\Lambda$-reduction, respectively.  $M \longrightarrow^* N$ means that there is
a sequence of reduction $\longrightarrow$ whose length is greater than or equal
to 0.

The relation $\longrightarrow_\beta$ represents ordinary $\beta$-reduction in
the \(\lambda\)-calculus; the relation $\longrightarrow_\blacklozenge$
represents that quotation $\TB_\alpha M$ is canceled by escape and $M$ is
spliced into the code fragment surrounding the escape; the relation
$\longrightarrow_\Lambda$ means that a stage abstraction applied to  stage $A$
reduces to the body of the abstraction where $A$ is substituted for the stage
variable.  There is no reduction rule for CSP as with Hanada and Igarashi
\cite{Hanada2014}.  The CSP operator $\%_\alpha$ disappears when $\varepsilon$
is substituted for $\alpha$.  We show an example of a reduction sequence below.
Underlines show the redexes.

\begin{align*}
    & \hspace{10mm} \underline{(\lambda f:\I\to\I.(\Lambda\alpha.\TB_\alpha (\%_\alpha f\ 1 + (\TBL_\alpha \TB_\alpha 3))\ \varepsilon))\ (\lambda x:\I.x)} \\
    & \longrightarrow_\beta (\Lambda\alpha.\TB_\alpha (\%_\alpha (\lambda x:\I.x)\ 1 + (\underline{\TBL_\alpha \TB_\alpha 3})))\ \varepsilon        \\
    & \longrightarrow_\blacklozenge \underline{(\Lambda\alpha.\TB_\alpha (\%_\alpha (\lambda x:\I.x)\ 1 + 3))\ \varepsilon}                                         \\
    & \longrightarrow_\Lambda \underline{(\lambda x:\I.x)\ 1} + 3                                                                                           \\
    & \longrightarrow_\beta 1 + 3                                                                                                                           \\
    & \longrightarrow^* 4
\end{align*}

\subsection{Type System}

In this section, we define the type system of \LMD.
It consists of eight judgment forms for signature well-formedness, type environment well-formedness, kind well-formedness, kinding, typing, kind equivalence, type equivalence, and term equivalence.
We list the judgment forms in Figure~\ref{fig:LMD-six-judgments}.
They are all defined in a mutual recursive manner.  We will discuss
each judgment below.

\begin{figure}[tbp]
  \begin{center}
    \begin{align*}
      \vdash & \Sigma                     & \text{signature well-formedness}        \\
      \V     & \G                         & \text{type environment well-formedness} \\
      \G     & \V K \iskind @ A           & \text{kind well-formedness}             \\
      \G     & \V \tau :: K @ A           & \text{kinding}                          \\
      \G     & \V M : \tau @ A            & \text{typing}                           \\
      \G     & \V K \E J @ A              & \text{kind equivalence}                 \\
      \G     & \V \tau \E \sigma :: K @ A & \text{type equivalence}                 \\
      \G     & \V M \E N : \tau @ A       & \text{term equivalence}
    \end{align*}
    \caption{Eight judgment forms of the type system of \LMD.}
    \label{fig:LMD-six-judgments}
  \end{center}
\end{figure}


\subsubsection{Signature and Type Environment Well-formedness.}
The rules for Well-formed signatures and type environments are
shown below:
%
{\small
\begin{center}
  \infrule{
  }{
    \vdash \emptyset
  }
  \hfil
  \infrule{
    \vdash \Sigma \andalso
    \V K \iskind @ \varepsilon \\
    X\notin\textit{dom}(\Sigma)
  }{
    \vdash \Sigma, X::K
  }
  \hfil
  \infrule{
    \vdash \Sigma \andalso
    \V \tau :: * @ \varepsilon \\
    c\notin\textit{dom}(\Sigma)
  }{
    \vdash \Sigma, c:\tau
  }
  \\[2mm]
  \infrule{
  }{
    \V \emptyset
  }
  \hfil
  \infrule{
    \V \Gamma \andalso
    \Gamma \V \tau :: * @ A \andalso
    x\notin\textit{dom}(\Sigma)
  }{
    \V \Gamma, x:\tau@A
  }
\end{center}
}

To add declarations to a signature, the kind/type of a (type-level)
constant has to be well-formed at stage \(\varepsilon\) so that it is
used at any stage.  In what follows, well-formedness is not explicitly
mentioned but we assume that all signatures and type environments are
well-formed.

\subsubsection{Kind Well-formedness and Kinding.}

The rules for kind well-formedness and kinding are a straightforward
adaptation from \LLF and \LTP, except for the following rule for type-level CSP.
\begin{center}
  \infrule[K-Csp]{
    \G \V \tau :: * @ A
  }{
    \G \V \tau :: * @ A\alpha
  }
\end{center}
Unlike the term level, type-level CSP is implicit because there is no staged
semantics for types.

\subsubsection{Typing.}
\label{sec:typing}

\begin{figure}[tbp]
  \begin{center}
    \infrule[\TConst]{c:\tau \in \Sigma}{\G \V c:\tau @A} \hfil
    \infrule[\TVar]{x:\tau @A \in \G}{\G \V x:\tau @A} \\[2mm]
    \infrule[\TAbs]{\G\V \sigma::*@A\andalso\G,x:\sigma@A\V M:\tau @A}{\G\V(\lambda (x:\sigma).M):(\Pi (x:\sigma).\tau)@A} \\[2mm]
    \infrule[\TApp]{\G\V M:(\Pi (x:\sigma).\tau)@A \andalso \G\V N:\sigma@A}{\G\V M\ N : \tau[x\mapsto N]@A} \\[2mm]
    \infrule[\TConv]{\G\V M:\tau @A \andalso \G\V \tau\equiv \sigma :: K@A}{\G\V M:\sigma@A} \\[2mm]
    \infrule[\TTB]{\G\V M:\tau @{A\alpha}}{\G\V\TB_{\alpha}M:\TW_{\alpha}\tau @A} \andalso
    \infrule[\TTBL]{\G\V M:\TW_{\alpha}\tau @A}{\G\V\TBL_{\alpha}M:\tau @{A\alpha}} \\[2mm]
    \infrule[\TGen]{\G\V M:\tau @A \andalso \alpha\notin\rm{FTV}(\G)\cup\rm{FTV}(A)}{\G\V\Lambda\alpha.M:\forall\alpha.\tau @A} \\[2mm]
    \infrule[\TIns]{\G\V M:\forall\alpha.\tau @A}{\G\V M\ B:\tau[\alpha \mapsto B]@A} \andalso
    \infrule[\TCsp]{\G\V M:\tau @A}{\G\V \%_\alpha M:\tau @{A\alpha}}
    \caption{Typing Rules.}
    \label{fig:typing-rules}
  \end{center}
\end{figure}

The typing rules of \LMD are shown in Figure~\ref{fig:typing-rules}.
The rule \TConst{} means that a constant can appear at any stage.
The rules \TVar, \TAbs, and \TApp{} are almost the same as those in the simply typed
lambda calculus or \LLF.  Additional conditions are that subterms must be
typed at the same stage (\TAbs{} and \TApp); the type
annotation/declaration on a variable has to be a proper type of kind
$*$ (\TAbs) at the stage where it is declared (\TVar{} and \TAbs).

% \TConv
As in standard dependent type systems, \TConv{} allows us to replace the type
of a term with an equivalent one. For example, assuming integers and
arithmetic, a value of type $\textrm{Vector}\ (4+1)$ can also have type
$\textrm{Vector}\ 5$ because of \TConv{}.

% Typing rules for a multi-stage calculus
The rules \TTB, \TTBL, \TGen, \TIns, and \TCsp{} are constructs for
multi-stage programming. \TTB{} and \TTBL{} are the same as in \LTP, as we
explained in Section \ref{sec:informal-overview}. The rule \TGen{} for stage
abstraction is straightforward. The condition
$\alpha\notin\rm{FTV}(\G)\cup\rm{FTV}(A)$ ensures that the scope of $\alpha$
is in $M$, and avoids capturing variables elsewhere. The
rule \TIns{} is for applications of stages to stage abstractions. The rule
\TCsp{} is for CSP, which means that, if term $M$ is of type $\tau$ at stage
$A$, then $\%_\alpha M$ is of type $\tau$ at stage $A\alpha$. Note that CSP
is also applied to the type \(\tau\) (although it is implicit) in the
conclusion. Thanks to implicit CSP, the typing rule is the same as in \LTP.


\subsubsection{Kind, Type and Term Equivalence.}

Since the syntax of kinds, types, and terms is mutually recursive,
the corresponding notions of equivalence are also mutually recursive.
They are congruences closed under a few axioms for term equivalence.
Thus, the rules for kind and type equivalences are not very interesting, 
except that implicit CSP is allowed.
We show a few representative rules below.

  {\small
    \begin{center}
      \infrule[\textsc{QK-Csp}]{%
        \G\V K \E J @ A
      }{
        \G\V K \E J @ A\alpha
      }
      \hfil
      \infrule[\QTCsp]{
        \G\V \tau \E \sigma :: *@A
      }{
        \G\V \tau \E \sigma :: *@{A\alpha}
      }
      \\[2mm]
      \infrule[\QTApp]{%
        \G\V \tau \E \sigma :: (\Pi x:\rho.K)@A \andalso
        \G\V M \E N : \rho @A
      }{
        \G\V \tau\ M \E \sigma\ N :: K[x \mapsto M]@A
      }
    \end{center}
  }

We show the rules for term equivalence  in
Figure~\ref{fig:term-equivalence-rules}, omitting straightforward rules for
reflexivity, symmetry, transitivity, and compatibility.  The rules \QBeta,
\QTBLTB, and \QLambda{} correspond to $\beta$-reduction,
$\blacklozenge$-reduction, and $\Lambda$-reduction, respectively.

\begin{figure}[tbp]
  \begin{center}
   \infrule[\QBeta]{\G,x:\sigma@A\V M:\tau @A \andalso \G\V N:\sigma@A}{\G\V(\lambda x:\sigma.M)\ N\E M[x\mapsto N] : \tau[x \mapsto N]@A} \\[2mm]
    \infrule[\QLambda]{\G\V (\Lambda\alpha.M) : \forall\alpha.\tau @A}{\G\V (\Lambda\alpha.M)\ \varepsilon \E M[\alpha \mapsto \varepsilon] : \tau[\alpha \mapsto \varepsilon]@A} \\[2mm]
    \infrule[\QTBLTB]{\G\V M \E N : \tau @A}{\G\V \TBL_\alpha(\TB_\alpha M) \E N : \tau @A} \hfil
    \infrule[\QPercent]{\G\V M:\tau @{A\alpha} \andalso \G\V M:\tau @A}{\G\V\%_\alpha M \E M : \tau @{A\alpha}}
    \caption{Term Equivalence Rules.}
    \label{fig:term-equivalence-rules}
  \end{center}
\end{figure}

% \QPercentの説明
The only rule that deserves elaboration is the last rule \QPercent.
Intuitively, it means that the CSP operator applied to term $M$ can be
removed if $M$ is also well-typed at the next stage \(A\alpha\).
For example, constants do not depend on the stage (see \TConst) and
so \(\G\V \%_\alpha c \E c : \tau @ A\alpha\) holds but variables
do depend on stages and so this rule does not apply.

\subsubsection{Example.}

We show an example of a dependently typed code generator in a
hypothetical language based on \LTP.  
% This rule is motivated by a practical consideration.
This language provides definitions by \textbf{let},
recursive functions (represented by \textbf{fix}), \textbf{if}-expressions,
and primitives cons, head, and tail to manipulate vectors. We assume that
$\text{cons}$ is of type $\Pi n:\I.\I \to \text{Vector}\ n \to \text{Vector}\ (n+1)$, 
$\text{head}$ is of type $\Pi n:\I.\text{Vector}\ (n+1) \to \I$, and
$\text{tail}$ is of type $\Pi n:\I.\text{Vector}\ (n+1) \to (\text{Vector}\ n)$.

Let's consider an application, for example, in computer graphics, in which we
have potentially many pairs of vectors of the fixed (but statically unknown)
length and a function---such as vector addition---to be applied to
them. This function should be fast because it is applied many times and be
safe because just one runtime error may ruin the whole long-running calculation.

\newcommand{\Vpn}{\text{Vector}\ (\%_\alpha n)}

Our goal is to define the function vadd of type
\[
  \Pi n:\I.\F\beta.\TW_\beta(\Vpn\to\Vpn\to\Vpn).
\]
\renewcommand{\Vpn}{\text{Vector}\ n}
It takes the length $n$ and returns ($\beta$-closed) code of a
function to add two vectors of length $n$.  The generated
code is run by applying it to \(\varepsilon\) to obtain
a function of type \(\Vpn\to\Vpn\to\Vpn\) as expected.

We start with the helper function vadd$_1$, which takes a stage, the length $n$ of vectors, and two quoted vectors as arguments and returns code that computes the addition of the given two vectors:
%
\begin{tabbing}
	  $\textbf{let}\ \text{vadd}_1 : \F\alpha.\Pi n:\I.\TW_\alpha\Vpn\to\TW_\alpha\Vpn\to\TW_\alpha\Vpn$                                \\
	  \hspace{6mm} \= $= \textbf{fix}\ f.\Lambda\alpha.\lambda n:\I.\ \lambda v_1:\TW_\alpha\Vpn.\ \lambda v_2:\TW_\alpha\Vpn.$            \\
	  \> \hspace{6mm} \= $\textbf{if}\ n = 0 \ \textbf{then}\ \TB_\alpha \text{nil}$ \\
	  \>\> $\textbf{else}\ \TB_\alpha ($ \= $\textbf{let}\ t_1 = \text{tail}\ (\TBL_\alpha v_1)\ \textbf{in}$ \\
	  \>\>\> $\textbf{let}\ t_2 = \text{tail}\ (\TBL_\alpha v_2)\ \textbf{in}$ \\
          \>\>\> $\text{cons}\ $\= $(\text{head}\ (\TBL_\alpha v_1) + \text{head} \ (\TBL_\alpha v_2))$ \\
          \>\>\>\> $\TBL_\alpha (f\ (n-1)\ (\TB_\alpha t_1)\ (\TB_\alpha t_2)))$
\end{tabbing}
Note that the generated code will not contain branching on $n$ or recursion.
(Here, we assume that the type system can determine whether $n=0$ when
\textbf{then}- and \textbf{else}-branches are typechecked so that
both branches can be given type \(\TW_\alpha \text{Vector }n\).)

Using vadd$_1$, the main function vadd can be defined as follows:
\renewcommand{\Vpn}{\text{Vector}\ (\%_\beta n)}
\begin{align*}
	  & \textbf{let}\ \text{vadd}: \Pi n:\I.\F\beta.\TW_\beta(\Vpn\to\Vpn\to\Vpn)                \\ 
	  & \hspace{6mm} = \lambda n:\I.\Lambda\beta.\TB_\beta (\lambda v_1:\Vpn.\ \lambda v_2:\Vpn. \\
	  & \hspace{63mm} \TBL_\beta (\text{vadd}_1\ \beta\ n\ (\TB_\beta\ v_1)\ (\TB_\beta\ v_2))) 
\end{align*}
\renewcommand{\Vpn}{\text{Vector}\ (\%_\beta 5)}%
The auxiliary function vadd$_1$ generates code to compute addition of
the formal arguments $v_1$ and $v_2$ without branching on $n$ or recursion.
As we mentioned already, if this function is applied to a
(nonnegative) integer constant, say 5, it returns function code for adding
two vectors of size 5.  The type of vadd\ 5, obtained by
substituting 5 for $n$, is
$\F\beta.\TW_\beta(\Vpn\to\Vpn\to\Vpn)$.
\renewcommand{\Vpn}{\text{Vector}\ 5}
If the obtained code is run by applying to \(\varepsilon\),
the type of vadd\ 5\ $\varepsilon$ is
\(\Vpn\to\Vpn\to\Vpn\) as expected.

There are other ways to implement the vector addition function:
by using tuples instead of lists if the length
for all the vectors is statically known or by checking dynamically the
lengths of lists for every pair.  However, our method is better than
these alternatives in two points.  First, our function, $\text{vadd}_1$
can generate functions for vectors of arbitrary length unlike the one
using tuples.  Second, $\text{vadd}_1$ has an advantage in speed over
the one using dynamic checking because it can generate an optimized
function for a given length.

We make two technical remarks before proceeding:
\begin{enumerate}
\item If the generated function code is composed with another piece of code of type, say,
\(\TW_\gamma \text{Vector }5\), \QPercent{} plays an essential role; that is,
\(\text{Vector }5\) and \(\text{Vector }(\%_\gamma 5)\), which would occur
by applying the generated code to \(\gamma\) (instead of \(\varepsilon\)), are syntactically
different types but \QPercent{} enables to equate them.
Interestingly, Hanada and Igarashi~\cite{Hanada2014} rejected the idea of
reduction that removes $\%_\alpha$ when they developed \LTP{}, as such
reduction does not match the operational behavior of the CSP operator in
implementations. However, as an equational system for multi-stage programs,
the rule \QPercent{} makes sense.
\item \renewcommand{\Vpn}{\text{Vector}\ n}
By using implicit type-level CSP, the type of vadd
could have been written
\(\Pi n:\I.\F\beta.\TW_\beta(\Vpn\to\Vpn\to\Vpn)\).  In this type,
Vector\ $n$ is given kind at stage \(\varepsilon\) and type-level CSP
implicitly lifts it to stage \(\beta\).  However, if a type-level
constant takes two or more arguments from different stages, term-level
CSP is necessary.  A matrix type (indexed by the numbers of columns
and rows) would be such an example.
\end{enumerate}

\subsection{Algorithmic Typing}

\AK{What should I call "normal typing" as?"}

% What is algorithmic typing?

Algorithmic typing rules are rules for implement type checker which takes an
environment \( \Gamma \), a signature \( \Sigma \), a term \( M \), a type \(
\tau \) and a stage \( A \) and returns \( \G \V M : \tau @ A \) or not. We
already defined typing rules for \LMD in Section \ref{sec:typing}, but they
include some non-syntax directed rules which their premise cannot be decided
from the goal. When we implement the type checker of \LMD, it is necessary to make
up type derivation trees from bottom to up. So, we need to make all typing rules
syntax-directed. And, of course, algorithmic typing is equivalent to the
original typing. We show the equivalence in Section \ref{sec:properties}.

% Judgements

Algorithmic typing is almost the same with normal typing rules threfore there
are 8 judegements as with normal typing in Figure \ref{fig:LMD-six-judgments}
e.g. well-formed signature, well-formed environment, well-formed kinding,
kinding, typing, kind equivalence, type equivalence, and term equivalence.  We
replace \( \V \) with \( \AV \) in judegements in order to distinguish them
from judegements of the normal typing and omit kinds and types from typing
equivalence and term equivalence, respectively. The definitions of well-formed
signature, well-formed environment, and well-formed kinding are the same with
original typing so we don't explain them here.

% Kinding rules and typing rules

Kinding rules and typing rules are the almost same as the normal typing but
there are no conversion rules, such as \TConv and \KConv. Conversion rules are
not syntax-directed because we can apply them any typing or kinding judgments.
As you can see in \cite{attapl}, all use of conversion rules in dependent type
system can be move to places of applications. This is because application rules
check the type of the function argument and the type of the argument are equal
but all other rules don't check the equality of types. So, the equivalence of
types become problem only in application rules. Then, if we replace type
equality check in application rules with conversion check, we can eliminate
conversion rules from algorithmic typing.

% Kind and type equivalence rules

It is the difference between kind and type equivalence rules of the normal
typing and the algorithmic typing is that there are no rules for transitivity,
symmetry, and reflexivity in the algorithmic typing. Instead of including them
in the rules, we prove transitivity, symmetry, and reflexivity as meta
theorems.

% Term equivalence rules -- especially on ANF

Second, We replaced all rules of term equivalence with \QAANF.

\subsection{Staged Semantics}

The reduction given above is full reduction and any redexes---even
under $\TB_\alpha$---can be reduced in an arbitrary order.
Following previous work~\cite{Hanada2014},
we introduce (small-step, call-by-value) staged semantics,
where only $\beta$-reduction or $\Lambda$-reduction at stage $\varepsilon$ or the outer-most $\blacklozenge$-reduction are allowed,
modeling an implementation.

We start with the definition of values. Since terms under quotations are
not executed, the grammar is indexed by stages.

\begin{definition}[Values]
  The family $V^A$ of sets of values, ranged over by $v^A$,
  is defined by the following grammar.  In the grammar, $A' \neq \varepsilon$ is assumed.
  \begin{align*}
    v^\varepsilon \in V^\varepsilon & ::= \lambda x:\tau.M \mid\ \TB_\alpha v^\alpha \mid \Lambda\alpha.v^\varepsilon                                             & \\
    v^{A'} \in V^{A'}               & ::= x \mid \lambda x:\tau.v^{A'} \mid v^{A'}\ v^{A'} \mid\ \TB_\alpha v^{A'\alpha} \mid \Lambda\alpha.v^{A'} \mid v^{A'}\ B & \\
                                    & \quad\   \mid\ \TBL_\alpha v^{A''} (\text{if } A' = A''\alpha \text{ for some } \alpha, A'' \neq \varepsilon)               & \\
                                    & \quad\   \mid\ \%_\alpha v^{A''} (\text{if } A' = A''\alpha)
  \end{align*}
\end{definition}

Values at stage $\varepsilon$ are $\lambda$-abstractions, quoted pieces of code,
or $\Lambda$-abstractions.  The body of a $\lambda$-abstraction can
be any term but the body of $\Lambda$-abstraction has to be a value.  It
means that the body of $\Lambda$-abstraction must be evaluated.  The
side condition for $\TBL_\alpha v^{A'}$ means that escapes in a value
can appear only under nested quotations because an escape under a
single quotation will splice the code value into the surrounding
code.  See Hanada and Igarashi~\cite{Hanada2014} for details.

In order to define staged reduction, we define redex and evaluation contexts.

\begin{definition}[Redex]
  The sets of $\varepsilon$-redexes (ranged over by $R^\varepsilon$) and $\alpha$-redexes (ranged over by $R^\alpha$) are defined by the following grammar.
  \begin{align*}
     & R^\varepsilon ::= (\lambda x:\tau.M)\ v^\varepsilon \mid (\Lambda\alpha.v^\varepsilon)\ \varepsilon \\
     & R^\alpha      ::=\ \TBL_\alpha \TB_\alpha v^\alpha                                                         \\
  \end{align*}
\end{definition}

\begin{definition}[Evaluation Context]
  Let $B$ be either \(\varepsilon\) or a stage variable \(\beta\).
  The family of sets $ECtx^A_B$ of evaluation contexts, ranged over by $E^A_B$, is defined by the following grammar (in which $A'$ stands for a non-empty stage).
  %  $A$ is assumed to be nonempty (but $B,A'$ can be empty).
  \begin{align*}
    E^\varepsilon_B \in ECtx^\varepsilon_B & ::= \square\ (\text{if\ } B = \varepsilon)
    \mid E^\varepsilon_B\ M \mid v^\varepsilon\ E^\varepsilon_B \mid \TB_\alpha E^\alpha_B
    \mid \Lambda\alpha.E^\varepsilon_B \mid E^\varepsilon_B\ A                                                                                    \\
    E^{A'}_B \in ECtx^{A'}_B               & ::= \square\ (\text{if } A' = B) \mid \lambda x:\tau.E^{A'}_B \mid E^{A'}_B\ M \mid v^{A'}\ E^{A'}_B \\
                                           & \mid \TB_\alpha E^{A'\alpha}_B \mid \TBL_\alpha E^{A}_B \ (\text{where } A\alpha = A')               \\
                                           & \mid \Lambda\alpha.E^{A'}_B \mid E^{A'}_B\ A \mid \%_\alpha\ E^{A}_B \ (\text{where } A\alpha = A')
  \end{align*}
\end{definition}

The subscripts $A$ and $B$ in $E^A_B$ stand for the stage of the evaluation
context and of the hole, respectively. The grammar represents that staged
reduction is left-to-right and call-by-value and terms under \(\Lambda\) are
reduced. Terms at non-$\varepsilon$ stages are not reduced, except
redexes of the form \(\TBL_\alpha \TB_\alpha v^\alpha\) at stage \(\alpha\).
A few examples of
evaluation contexts are shown below:
\begin{align*}
  \square\ (\lambda x:\I.x)                  & \in  ECtx^\varepsilon_\varepsilon \\
  \Lambda\alpha.\square\ \varepsilon            & \in ECtx^\varepsilon_\varepsilon  \\
  \TBL_\alpha \TB_\alpha \TBL_\alpha \square & \in ECtx^\alpha_\varepsilon
\end{align*}
%
We write $E^A_B[M]$ for the term obtained by filling the hole $\square$ in $E^A_B$ by $M$.

Now we define staged reduction using the redex and evaluation contexts.

\begin{definition}[Staged Reduction]\sloppy
  The staged reduction relation, written $M \longrightarrow_s N$, is defined by
  the least relation closed under the rules below.
  \begin{align*}
    E^A_\varepsilon [(\lambda x:\tau.M)\ v^\varepsilon] & \longrightarrow_s E^A_\varepsilon[M[x\mapsto v^\varepsilon]]      \\
    E^A_\varepsilon [(\Lambda\alpha.v^\varepsilon)\ A]  & \longrightarrow_s E^A_\varepsilon[v^\varepsilon[\alpha\mapsto A]] \\
    E^A_\alpha [\TBL_\alpha \TB_\alpha v^\alpha]        & \longrightarrow_s E^A_\alpha[v^\alpha]                            \\
  \end{align*}
\end{definition}

This reduction relation reduces a term in a deterministic,
left-to-right, call-by-value manner.  An application of an abstraction
is executed only at stage \(\varepsilon\) and only a quotation at
stage \(\varepsilon\) is spliced into the surrounding code---notice
that, if \(\TB_\alpha v^\alpha\) is at stage \(\varepsilon\), then the
redex \(\TBL_\alpha \TB_\alpha v^\alpha\) is at stage \(\alpha\).
In other words, terms in brackets are not evaluated until the terms are run
and arguments of a function are evaluated before the application.
We show an example of staged reduction.
Underlines show the redexes.
\begin{align*}
                    & (\Lambda\alpha.(\TB_\alpha \underline{\TBL_\alpha \TB_\alpha ((\lambda x:\I.x)\ 10))})\ \varepsilon \\
  \longrightarrow_s & \underline{(\Lambda\alpha.(\TB_\alpha ((\lambda x:\I.x)\ 10)))\ \varepsilon}                        \\
  \longrightarrow_s & \underline{(\lambda x:\I.x)\ 10}                                                                    \\
  \longrightarrow_s & 10
\end{align*}
